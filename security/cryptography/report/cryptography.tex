\documentclass[a4paper,14pt]{extarticle}

\usepackage[utf8x]{inputenc}
\usepackage[T1]{fontenc}
\usepackage[russian]{babel}
\usepackage{hyperref}
\usepackage{indentfirst}
\usepackage{here}
\usepackage{array}
\usepackage{graphicx}
\usepackage{caption}
\usepackage{subcaption}
\usepackage{chngcntr}
\usepackage{amsmath}
\usepackage{amssymb}
\usepackage[left=2cm,right=2cm,top=2cm,bottom=2cm,bindingoffset=0cm]{geometry}
\usepackage{multicol}
\usepackage{multirow}
\usepackage{titlesec}
\usepackage{listings}
\usepackage{color}
\usepackage{enumitem}
\usepackage{cmap}
\usepackage{underscore}

\definecolor{green}{rgb}{0,0.6,0}
\definecolor{gray}{rgb}{0.5,0.5,0.5}
\definecolor{purple}{rgb}{0.58,0,0.82}

\lstdefinelanguage{none}{}

\lstset{
	language={C++},
	inputpath={../generator/src/main/java/com/vaddya/hotelbooking},
	backgroundcolor=\color{white},
	commentstyle=\color{green},
	keywordstyle=\color{blue},
	numberstyle=\scriptsize\color{gray},
	stringstyle=\color{purple},
	basicstyle=\ttfamily\small,
	breakatwhitespace=false,
	breaklines=true,
	captionpos=b,
	keepspaces=true,
	numbers=left,
	numbersep=5pt,
	showspaces=false,
	showstringspaces=false,
	showtabs=false,
	tabsize=4,
	texcl=true,
	extendedchars=false,
	frame=single,
	morekeywords={IF, BIGSERIAL, SERIAL, TEXT, BIGINT, MONEY, BOOLEAN, REFERENCES}
}

\renewcommand{\le}{\ensuremath{\leqslant}}
\renewcommand{\leq}{\ensuremath{\leqslant}}
\renewcommand{\ge}{\ensuremath{\geqslant}}
\renewcommand{\geq}{\ensuremath{\geqslant}}
\renewcommand{\epsilon}{\ensuremath{\varepsilon}}
\renewcommand{\phi}{\ensuremath{\varphi}}
\renewcommand{\thefigure}{\arabic{figure}}
\newcommand{\code}[1]{\texttt{#1}}
\newcommand{\caret}{\^{}}

\titleformat*{\section}{\large\bfseries} 
\titleformat*{\subsection}{\normalsize\bfseries} 
\titleformat*{\subsubsection}{\normalsize\bfseries} 
\titleformat*{\paragraph}{\normalsize\bfseries} 
\titleformat*{\subparagraph}{\normalsize\bfseries} 

\counterwithin{figure}{section}
\counterwithin{equation}{section}
\counterwithin{table}{section}
\newcommand{\sign}[1][5cm]{\makebox[#1]{\hrulefill}}
\newcommand{\equipollence}{\quad\Leftrightarrow\quad}
\newcommand{\no}[1]{\overline{#1}}
\graphicspath{{../pics/}}
\captionsetup{justification=centering,margin=1cm}
\def\arraystretch{1.3}
\setlength\parindent{5ex}
\titlelabel{\thetitle.\quad}

\setitemize{topsep=0.3em, itemsep=0em}
\setenumerate{topsep=0.3em, itemsep=0em}

\begin{document}

\begin{titlepage}
\begin{center}
	Санкт-Петербургский Политехнический Университет Петра Великого\\[0.3cm]
	Институт компьютерных наук и технологий \\[0.3cm]
	Кафедра компьютерных систем и программных технологий\\[4cm]
	
	\textbf{ОТЧЕТ}\\ 
	\textbf{по лабораторной работе}\\[0.5cm]
	\textbf{<<Изучение прикладных протоколов в командной строке Linux>>}\\[0.1cm]
	Разработка сетевых приложений\\[3.0cm]
\end{center}

\begin{flushright}
	\begin{minipage}{0.45\textwidth}
		\textbf{Работу выполнил студент}\\[3mm]
		группа 43501/3 \hfill Дьячков В.В.\\[5mm]
		\textbf{Работу принял преподаватель}\\[5mm]
		\sign[3cm] \hfill Зозуля А.В. \\[5mm]
	\end{minipage}
\end{flushright}

\vfill

\begin{center}
	Санкт-Петербург\\[0.3cm]
	\the\year
\end{center}
\end{titlepage}

\addtocounter{page}{1}

\tableofcontents
\newpage

\section{Цель работы}

Исследовать возможности анализаторов трафика на примере бесплатного и свободно распространяемого инструмента WireShark.

\vspace{-0.5em}
\section{Программа работы}

\begin{enumerate}
	\item Создание GPG ключа на одном и импорт публичного ключа на другом устройстве.
	\item Подпись текстового сообщения на одном устройстве и проверка его подлинности на другом компьютере.
	\item Шифрование текстового сообщения на одном компьютере публичным ключом другого компьютера и его расшифровка.
\end{enumerate}

\vspace{-0.5em}
\section{Сведения о компьютерах}

Для вывода информации о используемых в процессе выполнения работы системах используем утилиту \code{uname}:
\lstinputlisting{uname1.txt}

\lstinputlisting{uname2.txt}

На компьютере №1 (\code{turing}) установлена ОС Ubuntu 18.04, а на компьютере №2 (\code{torvalds}) ОС Manjaro 18.0.

\vspace{-0.5em}
\section{Выполнение работы}

\subsection{PGP и GnuPG}

\textbf{PGP} (Pretty Good Privacy) -- компьютерная программа, также библиотека функций, позволяющая выполнять операции шифрования и цифровой подписи сообщений, файлов и другой информации, представленной в электронном виде, в том числе прозрачное шифрование данных на запоминающих устройствах, например, на жёстком диске.

К 1997 году PGP уже широко использовалась во всём мире и многие хотели создавать собственное ПО, совместимое с PGP 5. В 1997 году PGP Inc. предложила IETF-стандарт, названный \textbf{OpenPGP}. В IETF были созданы стандарты RFC 2440 (1998 год) и RFC 4880 (2007 год). В 1999 году силами Фонда свободного программного обеспечения (Free Software Foundation, FSF) была создана свободная реализация OpenPGP под названием GNU Privacy Guard.

\textbf{GnuPG} (GNU Privacy Guard, GPG) -- свободная программа для шифрования информации и создания электронных цифровых подписей. Разработана как альтернатива PGP и выпущена под свободной лицензией GNU General Public License. GPG может использоваться для симметричного шифрования, но в основном программа используется для ассиметричного шифрования информации.

Шифры используемые в GnuPG:

\begin{itemize}
	\item Симметричные шифры: IDEA, 3DES, CAST5, Blowfish, AES, AES192, AES256, Twofish
	\item Алгоритмы с открытым ключом: RSA, ElGamal, DSA
	\item Хэш-функции: MD5, SHA1, RIPEMD160, SHA256, SHA384, SHA512
\end{itemize}

\subsection{Генерация GPG ключей}

С помощью команды \code{gpg --full-generate-key} сгенерируем ключ:

\lstinputlisting{generate.txt}

Ключ длиной 4096 бит был сгенерирован с использованием алгоритма шифрования RSA. Был установлена дата истечения ключа: 10 июня 2019 года.

Для передачи публичного ключа второму устройству его необходимо экспортировать с помощью ключа \code{-{}-export}. Для наглядности воспользуемся опцией \code{-{}-armor} -- экспорт ключа в текстовом формате.

\lstinputlisting{export.txt}

Передадим этот файл второму устройству и импортируем с помощью ключа \code{-{}-import}.

\lstinputlisting{import.txt}

Видно, что ключ был успешно импортирован. После этого ключу был присвоен уровень доверия со значением 5 при помощи интерактивного меню.

\lstinputlisting{trust.txt}

\subsection{Подпись текстового сообщения}

На устройстве, на котором был сгенерирован ключ, создадим текстовый файл \code{message.txt} и подпишем его при помощи ключа \code{-{}-detach-sign}, который подписывает файл и сохраняет подпись в отдельный файл. Для наглядности оставим опцию \code{-{}-armor}. 

\lstinputlisting{sign.txt}

Передадим текстовый файл и сгенерированную подпись на второе устройство и проверим сначала исходный файл, а потом файл, в который были внесены изменения.

\lstinputlisting{verify.txt}

Видно, что файл, в котором не было изменений, был успешно проверен. После дозаписи в файл проверка проваливается. 

\subsection{Асимметричное шифрование текстового сообщения}

Создадим на втором устройстве текстовый файл \code{encrypted\_text.txt} и зашифруем его с помощью опции \code{-{}- encrypt} открытым ключом первого устройства. После ключа \code{-{}-recipient} указывается получатель сообщения. Для наглядности оставим ключ \code{-{}-armor}. 

\lstinputlisting{encrypt.txt}

Передадим зашифрованное сообщение на первое устройство и расшифруем его.

\lstinputlisting{decrypt.txt}

Видно, что исходное сообщение было успешно расшифровано с помощью закрытого ключа.

\section{Выводы}

В ходе выполнения данной работы:

\begin{itemize}
	\item получены навыки работы с утилитой \code{GnuPG};
	\item закреплены знания о несимметричном шифровании;
	\item рассмотрена на практике генерация и проверка электронной подписи;
	\item изучены возможности шифрования сообщений при помощи публичного ключа и дешифрации при помощи закрытого. 
\end{itemize}

\section*{Список использованных источников}

\begin{enumerate}
	\item Шифры используемые в GnuPG / OpenNET [Электронный ресурс]:\\
	URL: {\small\url{https://www.opennet.ru/docs/RUS/gph/ch02s07.html}}
	\item Используем GPG для шифрования сообщений и файлов / Хабр [Электронный ресурс]:\\
		URL: {\small\url{https://habr.com/post/358182/}}
	\item Exchanging keys \ The GNU Privacy Handbook [Электронный ресурс]:\\
		URL: {\small\url{https://www.gnupg.org/gph/en/manual/x56.html}}
\end{enumerate}

\section*{Дополнения к отчету}

\subsection*{Симметричное шифрование текстового сообщения}

Зашифруем сообщение утилитой GnuPG при помощи симметричного шифрования. Для этого добавим флаг \code{-{}-symmetric} и укажем алгоритм Blowfish при помощи флага \code{-{}-cipher-algo} (по умолчанию \code{AES256}). При шифровании была введена секретная фраза: <<pass>>. Для наглядности оставим ключ \code{-{}-armor}. 

\lstinputlisting{sym_encrypt.txt}

Передадим зашифрованное сообщение на второе устройство и расшифруем его, введя во всплывающее окно секретную фразу.

\lstinputlisting{sym_decrypt.txt}

Видно, что в первый раз был введен неверный секретный ключ, а во второй раз соощение было успешно расшифровано.

\end{document}
