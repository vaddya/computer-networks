\documentclass[a4paper,14pt]{extarticle}

\usepackage[utf8x]{inputenc}
\usepackage[T1]{fontenc}
\usepackage[russian]{babel}
\usepackage{hyperref}
\usepackage{indentfirst}
\usepackage{here}
\usepackage{array}
\usepackage{graphicx}
\usepackage{caption}
\usepackage{subcaption}
\usepackage{chngcntr}
\usepackage{amsmath}
\usepackage{amssymb}
\usepackage[left=2cm,right=2cm,top=2cm,bottom=2cm,bindingoffset=0cm]{geometry}
\usepackage{multicol}
\usepackage{multirow}
\usepackage{titlesec}
\usepackage{listings}
\usepackage{color}
\usepackage{enumitem}
\usepackage{cmap}
\usepackage{underscore}

\definecolor{green}{rgb}{0,0.6,0}
\definecolor{gray}{rgb}{0.5,0.5,0.5}
\definecolor{purple}{rgb}{0.58,0,0.82}

\lstdefinelanguage{none}{}

\lstset{
	language={C++},
	inputpath={../generator/src/main/java/com/vaddya/hotelbooking},
	backgroundcolor=\color{white},
	commentstyle=\color{green},
	keywordstyle=\color{blue},
	numberstyle=\scriptsize\color{gray},
	stringstyle=\color{purple},
	basicstyle=\ttfamily\small,
	breakatwhitespace=false,
	breaklines=true,
	captionpos=b,
	keepspaces=true,
	numbers=left,
	numbersep=5pt,
	showspaces=false,
	showstringspaces=false,
	showtabs=false,
	tabsize=4,
	texcl=true,
	extendedchars=false,
	frame=single,
	morekeywords={IF, BIGSERIAL, SERIAL, TEXT, BIGINT, MONEY, BOOLEAN, REFERENCES}
}

\renewcommand{\le}{\ensuremath{\leqslant}}
\renewcommand{\leq}{\ensuremath{\leqslant}}
\renewcommand{\ge}{\ensuremath{\geqslant}}
\renewcommand{\geq}{\ensuremath{\geqslant}}
\renewcommand{\epsilon}{\ensuremath{\varepsilon}}
\renewcommand{\phi}{\ensuremath{\varphi}}
\renewcommand{\thefigure}{\arabic{figure}}
\newcommand{\code}[1]{\texttt{#1}}
\newcommand{\caret}{\^{}}

\titleformat*{\section}{\large\bfseries} 
\titleformat*{\subsection}{\normalsize\bfseries} 
\titleformat*{\subsubsection}{\normalsize\bfseries} 
\titleformat*{\paragraph}{\normalsize\bfseries} 
\titleformat*{\subparagraph}{\normalsize\bfseries} 

\counterwithin{figure}{section}
\counterwithin{equation}{section}
\counterwithin{table}{section}
\newcommand{\sign}[1][5cm]{\makebox[#1]{\hrulefill}}
\newcommand{\equipollence}{\quad\Leftrightarrow\quad}
\newcommand{\no}[1]{\overline{#1}}
\graphicspath{{../pics/}}
\captionsetup{justification=centering,margin=1cm}
\def\arraystretch{1.3}
\setlength\parindent{5ex}
\titlelabel{\thetitle.\quad}

\setitemize{topsep=0.3em, itemsep=0em}
\setenumerate{topsep=0.3em, itemsep=0em}

\begin{document}

\begin{titlepage}
\begin{center}
	Санкт-Петербургский Политехнический Университет Петра Великого\\[0.3cm]
	Институт компьютерных наук и технологий \\[0.3cm]
	Кафедра компьютерных систем и программных технологий\\[4cm]
	
	\textbf{ОТЧЕТ}\\ 
	\textbf{по лабораторной работе}\\[0.5cm]
	\textbf{<<Изучение прикладных протоколов в командной строке Linux>>}\\[0.1cm]
	Разработка сетевых приложений\\[3.0cm]
\end{center}

\begin{flushright}
	\begin{minipage}{0.45\textwidth}
		\textbf{Работу выполнил студент}\\[3mm]
		группа 43501/3 \hfill Дьячков В.В.\\[5mm]
		\textbf{Работу принял преподаватель}\\[5mm]
		\sign[3cm] \hfill Зозуля А.В. \\[5mm]
	\end{minipage}
\end{flushright}

\vfill

\begin{center}
	Санкт-Петербург\\[0.3cm]
	\the\year
\end{center}
\end{titlepage}

\addtocounter{page}{1}

\tableofcontents
\newpage

\section{Техническое задание}

По каждому протоколу привести:

\begin{itemize}
	\item основные сведения о протоколе,
	\item описание основных команд,
	\item область применения и ограничения протокола,
	\item распечатку вывода консоли при выполнении задания,
	\item в выводах отразить, в чем заключается польза от выполнения данной работы.
\end{itemize}

\section{Прикладные протоколы}

\subsection{SMTP}

\subsubsection{Основные сведения о протоколе}

\subsubsection{Задание}

При помощи утилиты \code{openssl s_client} отправить email через SMTP-сервер Yandex.

\subsubsection{Выполнение задания}\ 

\lstinputlisting[caption=\code{smtp.log}]{smtp.log}

\subsection{POP3}

\subsubsection{Основные сведения о протоколе}

\subsubsection{Задание}

При помощи утилиты \code{openssl s_client} прочитать email при помощи POP3-сервера Yandex.

\subsubsection{Выполнение задания}\ 

\lstinputlisting[caption=\code{pop3.log}]{pop3.log}

\subsection{IMAP}

\subsubsection{Основные сведения о протоколе}

\subsubsection{Задание}

При помощи утилиты \code{openssl s_client} прочитать email при помощи IMAP-сервера Yandex.

\subsubsection{Выполнение задания}\ 

\lstinputlisting[caption=\code{imap.log}]{imap.log}

\subsection{HTTP}

\subsubsection{Основные сведения о протоколе}

\subsubsection{Задание}

При помощи утилиты \code{openssl s_client} прочитать главную страницу сайта \code{yandex.ru}.

\subsubsection{Выполнение задания}\ 

\lstinputlisting[caption=\code{http.log}]{http.log}

\subsection{FTP}

\subsubsection{Основные сведения о протоколе}

\subsubsection{Задание} 

\noindent При помощи утилиты \code{telnet} загрузить файл с FTP-сервера \code{ftp.sunet.se}.

\subsubsection{Выполнение задания}\ 

\lstinputlisting[caption=\code{ftp.log}]{ftp.log}

\subsection{TFTP}

\subsubsection{Основные сведения о протоколе}

\subsubsection{Задание}

При помощи утилиты \code{netcat} выяснить имя файла, который запрашивает удаленный хост (при помощи утилиты \code{tftp}) у данного хоста.

\subsubsection{Выполнение задания}\ 

\lstinputlisting[caption=\code{tftp.log}]{tftp.log}

\subsection{WebDAV}

\subsubsection{Основные сведения о протоколе}

\subsubsection{Задание}

При помощи утилиты \code{curl} загрузить файл, создать директорию, выгрузить файл на сервис \code{disk.yandex.ru}.

\subsubsection{Выполнение задания}\ 

\lstinputlisting[caption=\code{webdav.log}]{webdav.log}

\section{Выводы}

Логи выполнения заданий и отчет размещены в Git репозитории:\\ 
\url{https://github.com/vaddya/computer-networks}

\end{document}
