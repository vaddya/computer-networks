\include{settings}

\begin{document}

\begin{titlepage}
\begin{center}
	Санкт-Петербургский Политехнический Университет Петра Великого\\[0.3cm]
	Институт компьютерных наук и технологий \\[0.3cm]
	Кафедра компьютерных систем и программных технологий\\[4cm]
	
	\textbf{ОТЧЕТ}\\ 
	\textbf{по лабораторной работе}\\[0.5cm]
	\textbf{<<Анализ сетевого трафика>>}\\[0.1cm]
	Защита информации\\[3.0cm]
\end{center}

\begin{flushright}
	\begin{minipage}{0.5\textwidth}
		\textbf{Работу выполнил студент}\\[3mm]
		группа 43501/3 \hfill \sign[1.3cm] \hfill Дьячков В.В.\\[5mm]
		\textbf{Работу принял преподаватель}\\[5mm]
		\sign[1.3cm] \hfill к.т.н., доц. Новопашенный А.Г. \\[5mm]
	\end{minipage}
\end{flushright}

\vfill

\begin{center}
	Санкт-Петербург\\[0.3cm]
	\the\year
\end{center}
\end{titlepage}

\addtocounter{page}{1}

\tableofcontents
\newpage

\section{Техническое задание}

По каждому протоколу привести:

\begin{itemize}
	\item основные сведения о протоколе,
	\item описание основных команд,
	\item область применения и ограничения протокола,
	\item распечатку вывода консоли при выполнении задания,
	\item в выводах отразить, в чем заключается польза от выполнения данной работы.
\end{itemize}

\section{Прикладные протоколы}

\subsection{SMTP}

\subsubsection{Основные сведения о протоколе}

\subsubsection{Задание}

При помощи утилиты \code{openssl s_client} отправить email через SMTP-сервер Yandex.

\subsubsection{Выполнение задания}\ 

\lstinputlisting[caption=\code{smtp.log}]{smtp.log}

\subsection{POP3}

\subsubsection{Основные сведения о протоколе}

\subsubsection{Задание}

При помощи утилиты \code{openssl s_client} прочитать email при помощи POP3-сервера Yandex.

\subsubsection{Выполнение задания}\ 

\lstinputlisting[caption=\code{pop3.log}]{pop3.log}

\subsection{IMAP}

\subsubsection{Основные сведения о протоколе}

\subsubsection{Задание}

При помощи утилиты \code{openssl s_client} прочитать email при помощи IMAP-сервера Yandex.

\subsubsection{Выполнение задания}\ 

\lstinputlisting[caption=\code{imap.log}]{imap.log}

\subsection{HTTP}

\subsubsection{Основные сведения о протоколе}

\subsubsection{Задание}

При помощи утилиты \code{openssl s_client} прочитать главную страницу сайта \code{yandex.ru}.

\subsubsection{Выполнение задания}\ 

\lstinputlisting[caption=\code{http.log}]{http.log}

\subsection{FTP}

\subsubsection{Основные сведения о протоколе}

\subsubsection{Задание} 

\noindent При помощи утилиты \code{telnet} загрузить файл с FTP-сервера \code{ftp.sunet.se}.

\subsubsection{Выполнение задания}\ 

\lstinputlisting[caption=\code{ftp.log}]{ftp.log}

\subsection{TFTP}

\subsubsection{Основные сведения о протоколе}

\subsubsection{Задание}

При помощи утилиты \code{netcat} выяснить имя файла, который запрашивает удаленный хост (при помощи утилиты \code{tftp}) у данного хоста.

\subsubsection{Выполнение задания}\ 

\lstinputlisting[caption=\code{tftp.log}]{tftp.log}

\subsection{WebDAV}

\subsubsection{Основные сведения о протоколе}

\subsubsection{Задание}

При помощи утилиты \code{curl} загрузить файл, создать директорию, выгрузить файл на сервис \code{disk.yandex.ru}.

\subsubsection{Выполнение задания}\ 

\lstinputlisting[caption=\code{webdav.log}]{webdav.log}

\section{Выводы}

Логи выполнения заданий и отчет размещены в Git репозитории:\\ 
\url{https://github.com/vaddya/computer-networks}

\end{document}
