\documentclass[a4paper,14pt]{extarticle}

\usepackage[utf8x]{inputenc}
\usepackage[T1]{fontenc}
\usepackage[russian]{babel}
\usepackage{hyperref}
\usepackage{indentfirst}
\usepackage{here}
\usepackage{array}
\usepackage{graphicx}
\usepackage{caption}
\usepackage{subcaption}
\usepackage{chngcntr}
\usepackage{amsmath}
\usepackage{amssymb}
\usepackage[left=2cm,right=2cm,top=2cm,bottom=2cm,bindingoffset=0cm]{geometry}
\usepackage{multicol}
\usepackage{multirow}
\usepackage{titlesec}
\usepackage{listings}
\usepackage{color}
\usepackage{enumitem}
\usepackage{cmap}
\usepackage{underscore}

\definecolor{green}{rgb}{0,0.6,0}
\definecolor{gray}{rgb}{0.5,0.5,0.5}
\definecolor{purple}{rgb}{0.58,0,0.82}

\lstdefinelanguage{none}{}

\lstset{
	language={C++},
	inputpath={../generator/src/main/java/com/vaddya/hotelbooking},
	backgroundcolor=\color{white},
	commentstyle=\color{green},
	keywordstyle=\color{blue},
	numberstyle=\scriptsize\color{gray},
	stringstyle=\color{purple},
	basicstyle=\ttfamily\small,
	breakatwhitespace=false,
	breaklines=true,
	captionpos=b,
	keepspaces=true,
	numbers=left,
	numbersep=5pt,
	showspaces=false,
	showstringspaces=false,
	showtabs=false,
	tabsize=4,
	texcl=true,
	extendedchars=false,
	frame=single,
	morekeywords={IF, BIGSERIAL, SERIAL, TEXT, BIGINT, MONEY, BOOLEAN, REFERENCES}
}

\renewcommand{\le}{\ensuremath{\leqslant}}
\renewcommand{\leq}{\ensuremath{\leqslant}}
\renewcommand{\ge}{\ensuremath{\geqslant}}
\renewcommand{\geq}{\ensuremath{\geqslant}}
\renewcommand{\epsilon}{\ensuremath{\varepsilon}}
\renewcommand{\phi}{\ensuremath{\varphi}}
\renewcommand{\thefigure}{\arabic{figure}}
\newcommand{\code}[1]{\texttt{#1}}
\newcommand{\caret}{\^{}}

\titleformat*{\section}{\large\bfseries} 
\titleformat*{\subsection}{\normalsize\bfseries} 
\titleformat*{\subsubsection}{\normalsize\bfseries} 
\titleformat*{\paragraph}{\normalsize\bfseries} 
\titleformat*{\subparagraph}{\normalsize\bfseries} 

\counterwithin{figure}{section}
\counterwithin{equation}{section}
\counterwithin{table}{section}
\newcommand{\sign}[1][5cm]{\makebox[#1]{\hrulefill}}
\newcommand{\equipollence}{\quad\Leftrightarrow\quad}
\newcommand{\no}[1]{\overline{#1}}
\graphicspath{{../pics/}}
\captionsetup{justification=centering,margin=1cm}
\def\arraystretch{1.3}
\setlength\parindent{5ex}
\titlelabel{\thetitle.\quad}

\setitemize{topsep=0.3em, itemsep=0em}
\setenumerate{topsep=0.3em, itemsep=0em}

\begin{document}

\begin{titlepage}
\begin{center}
	Санкт-Петербургский Политехнический Университет Петра Великого\\[0.3cm]
	Институт компьютерных наук и технологий \\[0.3cm]
	Кафедра компьютерных систем и программных технологий\\[4cm]
	
	\textbf{ОТЧЕТ}\\ 
	\textbf{по лабораторной работе}\\[0.5cm]
	\textbf{<<Изучение прикладных протоколов в командной строке Linux>>}\\[0.1cm]
	Разработка сетевых приложений\\[3.0cm]
\end{center}

\begin{flushright}
	\begin{minipage}{0.45\textwidth}
		\textbf{Работу выполнил студент}\\[3mm]
		группа 43501/3 \hfill Дьячков В.В.\\[5mm]
		\textbf{Работу принял преподаватель}\\[5mm]
		\sign[3cm] \hfill Зозуля А.В. \\[5mm]
	\end{minipage}
\end{flushright}

\vfill

\begin{center}
	Санкт-Петербург\\[0.3cm]
	\the\year
\end{center}
\end{titlepage}

\addtocounter{page}{1}

\tableofcontents
\newpage

\section{Техническое задание}

По каждому протоколу привести:

\begin{itemize}
	\item основные сведения о протоколе,
	\item описание основных команд,
	\item область применения и ограничения протокола,
	\item распечатку вывода консоли при выполнении задания,
	\item в выводах отразить, в чем заключается польза от выполнения данной работы.
\end{itemize}

\section{Прикладные протоколы}

\subsection{SMTP}

\subsubsection{Основные сведения о протоколе}

SMTP (Simple Mail Transfer Protocol) -- это широко используемый сетевой протокол, предназначенный для передачи электронной почты в сетях TCP/IP.

Основные команды:

\begin{itemize}
	\item \code{EHLO <server name>}. Диалог клиента и сервера, как правило, начинается с приветствия. В RFC 821 в качестве приветствия предлагалась команда \code{HELO}. Однако с введением расширений \code{ESMTP}, эта команда была заменена на \code{EHLO}.
	\item \code{MAIL FROM <address>}. С помощью этой команды серверу сообщается адрес отправителя письма. На этот адрес письмо должно вернуться в случае невозможности доставки.
	\item \code{RCPT TO <address>}. Доставка сообщения возможна, только если указан хотя бы один доступный адрес получателя. Команда \code{RCPT} принимает в качестве аргумента только один адрес. Если нужно послать письмо большему числу адресатов, то команду \code{RCPT} следует повторять для каждого.
	\item \code{DATA <data>}. C помощью этой команды серверу передается текст сообщения, состоящий из заголовка и отделенного от него пустой строкой тела сообщения.
	\item \code{QUIT}. Командой QUIT клиент заканчивает диалог с сервером.
\end{itemize}

\subsubsection{Задание}

При помощи утилиты \code{openssl s_client} отправить email через SMTP-сервер Yandex.

\subsubsection{Выполнение задания}

\lstinputlisting[caption=\code{smtp.log}]{smtp.log}

\subsection{POP3}

\subsubsection{Основные сведения о протоколе}

POP3 (Post Office Protocol Version 3) -- стандартный интернет-протокол прикладного уровня, используемый клиентами электронной почты для получения почты с удалённого сервера по TCP-соединению.

Основные команды:

\begin{itemize}
	\item \code{USER <username>}. Передаёт серверу имя пользователя.
	\item \code{PASS <password>}. Передаёт серверу пароль почтового ящика.
	\item \code{RETR <number>}. Сервер передаёт сообщение с указанным номером.
	\item \code{TOP <number> <lines>}. Сервер возвращает заголовки указанного сообщения, пустую строку и указанное количество первых строк тела сообщения.
	\item \code{QUIT}. Закрытие соединения с сервером. 
\end{itemize}

\subsubsection{Задание}

При помощи утилиты \code{openssl s_client} прочитать email при помощи POP3-сервера Yandex.

\subsubsection{Выполнение задания}

\lstinputlisting[caption=\code{pop3.log}]{pop3.log}

\subsection{IMAP}

\subsubsection{Основные сведения о протоколе}

IMAP (Internet Message Access Protocol) -- протокол прикладного уровня для доступа к электронной почте. Базируется на транспортном протоколе TCP и использует порт 143, а IMAPS (IMAP поверх SSL) — порт 993. IMAP работает только с сообщениями и не требует каких-либо пакетов со специальными заголовками.

Основные команды:

\begin{itemize}
	\item \code{CAPABILITY}. Запрос списка возможностей
	\item \code{LOGOUT}. Завершает сеанс для текущего идентификатора пользователя.
	\item \code{LOGIN <username> <password>}. Позволяет клиенту при регистрации на сервере IMAP использовать идентификатор пользователя и пароль в обычном текстовом виде.
	\item \code{AUTHENTICATE <method>}. Позволяет клиенту использовать при регистрации на сервере IMAP альтернативные методы проверки подлинности. 
	\item \code{SELECT <mailbox>}. Открывает доступ к указанному почтовому ящику.
	\item \code{EXAMINE <mailbox>}. Аналогично команде SELECT , но почтовый ящик открывается только для чтения.
	\item \code{LIST <path> <mailbox>}. Возвращает список каталогов и почтовых ящиков, соответствующих указанным аргументам.
	\item \code{STATUS <mailbox>}. Выдача состояния ящика
	\item \code{APPEND <mailbox> [<flags>]}. Добавляет сообщение в конец указанного почтового ящика.
	\item и другие.
\end{itemize}

\subsubsection{Задание}

При помощи утилиты \code{openssl s_client} прочитать email при помощи IMAP-сервера Yandex.

\subsubsection{Выполнение задания}

\lstinputlisting[caption=\code{imap.log}]{imap.log}

\subsection{HTTP}

\subsubsection{Основные сведения о протоколе}

HTTP (HyperText Transfer Protocol) -- протокол прикладного уровня передачи данных изначально -- в виде гипертекстовых документов в формате «HTML», в настоящий момент используется для передачи произвольных данных. 

Формат HTTP-запроса:

\begin{itemize}
	\item \code{<Request-line>} -- строка запроса
	\item \code{<General-header>} -- общий заголовок
	\item \code{<Request-header>} -- заголовок запроса
	\item \code{<Entity-header>} -- заголовок сообщения
	\item \code{<Body>} -- тело запроса
\end{itemize}

Методы HTTP-запроса:

\begin{itemize}
	\item GET
	\item POST
	\item HEAD
	\item PUT
	\item DELETE
	\item OPTIONS
	\item UPDATE
\end{itemize}

\subsubsection{Коды ответов}
\begin{itemize}
	\item 1xx -- Информационные
	\item 2xx -- ОК
	\item 3xx -- Переадресация
	\item 4xx -- Ошибка клиента
	\item 5xx -- Ошибка сервера
\end{itemize}

\subsubsection{Задание}

При помощи утилиты \code{openssl s_client} прочитать главную страницу сайта \code{yandex.ru}.

\subsubsection{Выполнение задания}

\lstinputlisting[caption=\code{http.log}]{http.log}

\subsection{FTP}

\subsubsection{Основные сведения о протоколе}

FTP (File Transfer Protocol) -- является широко используемым протоколом для обмена файлами по любой сети, поддерживающей протокол TCP/IP. FTP поддерживает два режима передачи.

Активный режим:

\begin{itemize}
	\item Режим <<по умолчанию>>.
	\item Сервер инициирует соединение данных.
	\item Клиент открывает слушающий порт.
	\item Номер TCP-порта сервера – 20.
	\item Невозможно использовать с технологиями типа NAT, Proxy.
	\item Обычно запрещён в межсетевых экранах.
\end{itemize}

Пассивный режим:

\begin{itemize}
	\item Клиент инициирует соединение данных.
	\item Сервер информирует о параметрах канала данных.
	\item Сервер открывает слушающий порт.
	\item Поддерживается не всеми реализациями.
\end{itemize}

\subsubsection{Основные команды}
\begin{itemize}
	\item \code{USER <username>}. Имя пользователя для входа на сервер.
	\item \code{PASS <password>}. Пароль.
	\item \code{ABOR}. Прервать передачу файла.
	\item \code{QUIT}. Отключиться.
	\item \code{DELE <filename>}. Удалить файл.
	\item \code{PWD}. Возвращает текущий каталог.
	\item \code{PORT a1, a2, a3, a4, p1, p2}. Войти в активный режим, порт p1 * 256 + p2.
	\item \code{PASV}. Войти в пассивный режим.
	\item \code{RETR <filename>}. Скачать файл.
	\item \code{STOR <filename>}. Загрузить файл.
	\item \code{LIST [<path>]}. Получить список файлов каталога.
	\item \code{NLST [<path>]}. Получить список имён файлов каталога.
\end{itemize}

\subsubsection{Задание} 

\noindent При помощи утилиты \code{telnet} загрузить файл с FTP-сервера \code{ftp.sunet.se}.

\subsubsection{Выполнение задания}

\lstinputlisting[caption=\code{ftp.log}]{ftp.log}

\subsection{TFTP}

\subsubsection{Основные сведения о протоколе}

TFTP (Trivial File Transfer Protocol) -- используется главным образом для первоначальной загрузки бездисковых рабочих станций. TFTP, в отличие от FTP, не содержит возможностей аутентификации (хотя возможна фильтрация по IP-адресу) и основан на транспортном протоколе UDP.

Сначала в TFTP-пакете идет поле размером в 2 байта, определяющее тип пакета:

\begin{itemize}
	\item Read Request (RRQ, \#1) -- запрос на чтение файла.
	\item Write Request (WRQ, \#2) -- запрос на запись файла.
	\item Data (DATA, \#3) -- данные, передаваемые через TFTP.
	\item Acknowledgment (ACK, \#4) -- подтверждение пакета.
	\item Error (ERR, \#5) -- ошибка.
	\item Option Acknowledgment (OACK, \#6) -- подтверждение опций.
\end{itemize}

В TFTP существует 2 режима передачи (режим Mail, определенный в IEN 133, признан устаревшим):

\begin{itemize}
	\item \code{netascii} -- файл перед передачей перекодируется в ASCII.
	\item \code{octet} -- файл передается без изменений.
\end{itemize}

\subsubsection{Задание}

При помощи утилиты \code{netcat} выяснить имя файла, который запрашивает удаленный хост (при помощи утилиты \code{tftp}) у данного хоста.

\subsubsection{Выполнение задания}

\lstinputlisting[caption=\code{tftp.log}]{tftp.log}

\subsection{WebDAV}

\subsubsection{Основные сведения о протоколе}

WebDAV (Web Distributed Authoring and Versioning), DAV — набор расширений и дополнений к протоколу HTTP, поддерживающих совместную работу пользователей над редактированием файлов и управление файлами на удаленных веб-серверах. Кроме того, DAV широко применяется в качестве протокола для доступа через Интернет и манипулирования содержимым систем документооборота.

WebDAV расширяет HTTP следующими методами запроса:

\begin{itemize}
	\item \code{PROPFIND} -- получение свойств объекта на сервере в формате XML. Также можно получать структуру репозитория (дерево каталогов);
	\item \code{PROPPATCH} -- изменение свойств за одну транзакцию;
	\item \code{MKCOL} -- создать коллекцию объектов (каталог в случае доступа к файлам);
	\item \code{COPY} -- копирование из одного URI в другой;
	\item \code{MOVE} -- перемещение из одного URI в другой;
	\item \code{LOCK} -- поставить блокировку на объекте. WebDAV поддерживает эксклюзивные и общие (shared) блокировки;
	\item \code{UNLOCK} -- снять блокировку с ресурса.
\end{itemize}

\subsubsection{Задание}

При помощи утилиты \code{curl} загрузить файл, создать директорию, выгрузить файл на сервис \code{disk.yandex.ru}.

\subsubsection{Выполнение задания}

\lstinputlisting[caption=\code{webdav.log}]{webdav.log}

\section{Выводы}

В процессе выполнения данной работы изучены протоколы и рассмотрено их применение на практике:

\begin{itemize}
	\item почтовые протокола (SMTP, POP3, IMAP);
	\item протокол передачи файлов и объектов HTTP;
	\item протоколы передачи файлов FTP и TFTP;
	\item протокол WebDAV для совместной работы над файлами поверх протокола HTTP. 
\end{itemize}

Логи выполнения заданий и отчет размещены в Git репозитории:\\ 
\url{https://github.com/vaddya/computer-networks}

\end{document}
