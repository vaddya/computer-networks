\documentclass[a4paper,14pt]{extarticle}

\usepackage[utf8x]{inputenc}
\usepackage[T1]{fontenc}
\usepackage[russian]{babel}
\usepackage{hyperref}
\usepackage{indentfirst}
\usepackage{here}
\usepackage{array}
\usepackage{graphicx}
\usepackage{caption}
\usepackage{subcaption}
\usepackage{chngcntr}
\usepackage{amsmath}
\usepackage{amssymb}
\usepackage[left=2cm,right=2cm,top=2cm,bottom=2cm,bindingoffset=0cm]{geometry}
\usepackage{multicol}
\usepackage{multirow}
\usepackage{titlesec}
\usepackage{listings}
\usepackage{color}
\usepackage{enumitem}
\usepackage{cmap}
\usepackage{underscore}

\definecolor{green}{rgb}{0,0.6,0}
\definecolor{gray}{rgb}{0.5,0.5,0.5}
\definecolor{purple}{rgb}{0.58,0,0.82}

\lstdefinelanguage{none}{}

\lstset{
	language={C++},
	inputpath={../generator/src/main/java/com/vaddya/hotelbooking},
	backgroundcolor=\color{white},
	commentstyle=\color{green},
	keywordstyle=\color{blue},
	numberstyle=\scriptsize\color{gray},
	stringstyle=\color{purple},
	basicstyle=\ttfamily\small,
	breakatwhitespace=false,
	breaklines=true,
	captionpos=b,
	keepspaces=true,
	numbers=left,
	numbersep=5pt,
	showspaces=false,
	showstringspaces=false,
	showtabs=false,
	tabsize=4,
	texcl=true,
	extendedchars=false,
	frame=single,
	morekeywords={IF, BIGSERIAL, SERIAL, TEXT, BIGINT, MONEY, BOOLEAN, REFERENCES}
}

\renewcommand{\le}{\ensuremath{\leqslant}}
\renewcommand{\leq}{\ensuremath{\leqslant}}
\renewcommand{\ge}{\ensuremath{\geqslant}}
\renewcommand{\geq}{\ensuremath{\geqslant}}
\renewcommand{\epsilon}{\ensuremath{\varepsilon}}
\renewcommand{\phi}{\ensuremath{\varphi}}
\renewcommand{\thefigure}{\arabic{figure}}
\newcommand{\code}[1]{\texttt{#1}}
\newcommand{\caret}{\^{}}

\titleformat*{\section}{\large\bfseries} 
\titleformat*{\subsection}{\normalsize\bfseries} 
\titleformat*{\subsubsection}{\normalsize\bfseries} 
\titleformat*{\paragraph}{\normalsize\bfseries} 
\titleformat*{\subparagraph}{\normalsize\bfseries} 

\counterwithin{figure}{section}
\counterwithin{equation}{section}
\counterwithin{table}{section}
\newcommand{\sign}[1][5cm]{\makebox[#1]{\hrulefill}}
\newcommand{\equipollence}{\quad\Leftrightarrow\quad}
\newcommand{\no}[1]{\overline{#1}}
\graphicspath{{../pics/}}
\captionsetup{justification=centering,margin=1cm}
\def\arraystretch{1.3}
\setlength\parindent{5ex}
\titlelabel{\thetitle.\quad}

\setitemize{topsep=0.3em, itemsep=0em}
\setenumerate{topsep=0.3em, itemsep=0em}

\begin{document}

\begin{titlepage}
\begin{center}
	Санкт-Петербургский Политехнический Университет Петра Великого\\[0.3cm]
	Институт компьютерных наук и технологий \\[0.3cm]
	Кафедра компьютерных систем и программных технологий\\[4cm]
	
	\textbf{ОТЧЕТ}\\ 
	\textbf{по лабораторной работе}\\[0.5cm]
	\textbf{<<Изучение прикладных протоколов в командной строке Linux>>}\\[0.1cm]
	Разработка сетевых приложений\\[3.0cm]
\end{center}

\begin{flushright}
	\begin{minipage}{0.45\textwidth}
		\textbf{Работу выполнил студент}\\[3mm]
		группа 43501/3 \hfill Дьячков В.В.\\[5mm]
		\textbf{Работу принял преподаватель}\\[5mm]
		\sign[3cm] \hfill Зозуля А.В. \\[5mm]
	\end{minipage}
\end{flushright}

\vfill

\begin{center}
	Санкт-Петербург\\[0.3cm]
	\the\year
\end{center}
\end{titlepage}

\addtocounter{page}{1}

\tableofcontents
\newpage

\section{Техническое задание}

% Раздел «Техническое задание» должен содержать подробное описание задания на разработку. Здесь следует указать все требования, предъявляемые к разрабатываемым приложениям, описать накладываемые ограничения.

\textbf{Система передачи файлов}.

\begin{itemize}
	\item \textbf{Задание:} разработать приложение-клиент и приложение сервер, обеспечивающие функции обмена файлами.

	\item \textbf{Серверное} приложение должно реализовывать следующие функции:
	\begin{enumerate}
		\item Прослушивание определенного порта
		\item Обработка запросов на подключение по этому порту от клиентов
		\item Поддержка одновременной работы нескольких клиентов через механизм нитей
		\item Приём файла от клиента
		\item Передача по запросу клиента списка файлов текущего каталога
		\item Приём запросов на передачу файла и передача файла клиенту
		\item Навигация по системе каталогов
		\item Обработка запроса на отключение клиента
		\item Принудительное отключение клиента
	\end{enumerate}

	\item \textbf{Клиентское} приложение должно реализовывать следующие функции:
	\begin{enumerate}
		\item Установление соединения с сервером
		\item Получение от сервера списка файлов каталога
		\item Операции навигации по системе каталогов
		\item Передача файла серверу
		\item Приём файла от сервера
		\item Разрыв соединения
		\item Обработка ситуации отключения клиента сервером
	\end{enumerate}

	\item \textbf{Настройки приложений.} Разработанное клиентское приложение должно предоставлять пользователю настройку IP-адреса или доменного имени файл-сервера и номера порта, используемого сервером.
	
	Разработанное серверное приложение должно предоставлять пользователю настройку корневого каталога для клиентских приложений.

	\item \textbf{Методика тестирования.} Для тестирования приложений запускается файловый сервер и несколько клиентов. В процессе тестирования проверяются основные возможности приложения по передаче файлов и навига	ции по системе каталогов.

\end{itemize}

\section{Анализ задания и выбор способа решения}

% В разделе «Анализ задания и выбор способа решения» проводится анализ поставленного технического задания, выбирается метод (способ, технология, методика) решения. В этом же разделе описываются те технологии, протоколы, библиотеки, которые будут использоваться для решения поставленных задач.

\section{Разработка сетевых приложений}

% Раздел «Разработка сетевых приложений» должен содержать описание архитектуры и логики взаимодействия компонентов сетевого приложения. Необходимо специфицировать разработанные прикладные протоколы, показать режимы работы всей сетевой системы. Особенности сетевого взаимодействия следует подкрепить иллюстративными материалами: структурными схемами, диаграммами.

\section{Описание разработанных приложений}

% В разделе «Описание разработанных приложений» следует описать архитектуру, логику функционирования и особенности реализации приложений. Здесь необходимо изобразить структуру программы, взаимосвязь ее компонентов, описать логику функционирования каждого из имеющихся режимов работы, дать краткое описание ключевых классов, методов, процедур, функций и потоков программы.

\section{Тестирование приложений и анализ результатов}

% Описание проведённых процедур тестирования разработанных приложений приводится в разделе «Тестирование приложений и анализ результатов».

\section{Выводы}

% Раздел «Выводы» должен включать в себя содержательные выводы, сделанные именно по данной работе. Здесь следует подчеркнуть особенности используемых протоколов, режимов работы, применяемых технологий и библиотек.

\newpage

\section*{Приложение}

% В «Приложениях» приводятся прокомментированные исходные тексты разработанных во время лабораторных работ программ. Программы должны быть распечатаны моноширинным шрифтом (например, Courier, кегль– 9).

\end{document}
